\section{前言\label{preface}}

集成电路设计自上世纪七八十年代以来,产业得到了极大的发展。随着信息技术的普及,集成电路作为信息产业的支柱产业之一,对人员的需求,对人员知识水平的要求逐渐提高。
为集成电路的发展,各种工具应运而生,然而国内设计人员对知识的理解,对工具的学习仍然有相当的差距,远不能支撑国内对高端人才的渴求。

人的知识需要成为体系,方能支撑各种可能的工作需求。

时序信息文件作为集成电路设计中标准单元/宏单元设计人员与物理后端设计人员之间重要的数据接口,是为双方设计人员所不熟悉的中间灰色地点。
实际工作过程中,很多宏单元设计人员不重视该部分信息的准确性和完备性,而物理后端设计人员对该部分信息同样理解不足,仅知道工具所识别部分,不能完整的识别和判断其中信息的准确性,从而造成信息传递过程中的失真,最终导致流片后芯片的功能异常、良率低下等问题。

本书结合自己多年的工作经验,总结工作中所接触过的相关的时序信息、所涉及的时序信息提取方式。
希望能够帮助集成电路工作者开展相关的设计工作,以期更多的人能够在集成电路设计道路上走的更远。

~\\

~\\

\hfill 丁健平

\hfill 2019年夏\ 于\ 北京海淀
